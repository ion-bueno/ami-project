\section{Conclusion}

From the integrated sector trends, we can see a drop in \co emissions. This can be directly attributed to the manifold effects of the pandemic. 

We tried to quantify the effect and correlate emission drops to active cases of COVID-19. On a time-series, we could not find a direct relation, especially not for all countries. We thus decided to look at the integrated emission drops and case numbers/deaths. However, one can not deduce a general trend from these two plots either. We therefore conclude that we have to discuss each country by its own.

Still, we have to answer our research question with these results. As we can see from the bubble plot of the Paris climate goals, most countries will miss the \(1.5^\circ C\) goal. We want to estimate from the plot how much the emission drop helps each country to reach its goal.

Due to the pandemic, most countries see an emission drop by over 15\% for the first half of 2020. Of course, we cannot say much about how the \co emissions will behave in the next ten years and quantifying the effect of the pandemic on countries reaching their climate goals is difficult. We want to do these calculations exemplary with Japan, as it is the country with highest gap to the \(1.5^\circ C\) goal. We see that Japan would need to further reduce its emissions by roughly 15\%. This means, that Japan needs to reduce its emissions on average by 15\% every year. We see from the total average emission drop plot that Japan reduced its emission by 22.5\% during the first half of 2020. This is just slightly more than Japan would need to reduce the emissions at least. We can do the same assessment for every country. We conclude that for every country, the emissions would need to stay permanently on a similar level as in the first half of 2020, every year, until 2030.




