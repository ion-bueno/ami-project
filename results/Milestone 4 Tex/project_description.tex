\section{Project Description}

% What we try to convey: What conclusions can we draw from an answer to the research question? What are the limitations? 


\section*{Research Question}
During the pandemic, a lot of countries went through a lock down, brought the industry to a halt, limited mobility. Only workers with the most necessary jobs were allowed to work outside their home. It is a fair assumption that this will lead to a reduction in greenhouse gas emissions. Most countries of the world signed international climate agreement and promised to contribute to fighting the climate crisis. Now of course it would be interesting to quantify the reduction in \co emissions and see how much it contributes to each countries climate goals as stated in international agreements. We thus formulate the research question as follows:
\vspace{1em}
\begin{center}
\textit{\large To what extent will the COVID-19 pandemic contribute towards reaching goals stated in international agreements?}
\end{center}
\vspace{1em}
We want to further specify our research question and investigate the goals of the Paris climate agreement. This is not a limitation actually, because the agreement also incorporates other international agreements, like the EU Green Deal. 

\section*{Limitations and Scope}
We only focus on \co emissions here, as this the greenhouse gas with the biggest impact on earth's climate.
To further reduce the complexity of the project, we limit the scope to the eight main \co emitting countries. This is mainly due to the workload we would have to invest in preprocessing climate goal data for every country of the world, as this cannot be easily automated. But we chose these eight countries carefully and ensured that we cover more than two thirds of the global \co emissions.

\section*{Goals of this analysis}

To highlight what we work on in this project, we summarized our main goals.

\begin{itemize}
	\item Analyze dataset with country and sectors resolved \co emissions.
	\item Predict 2020 emissions from previous data (not considering the event of the corona crisis).
	\item Identify and collect indicators for each sector.
	\item Apply a machine learning model to predict the real \co emissions in 2020.
	\item Extract the \co emission drop from the difference of the predictions.
	\item Establish an indicator for the severity of COVID-19 in a country.
	\item Find correlations with the severity of COVID-19 in a country.
\end{itemize}

\section*{Important questions for politicians}

We also summarized the most important questions for politicians and tried to answer them at the end. We think that with these questions answered, our project generates new and valuable insights on how to tackle the climate crisis.

\begin{itemize}
	\item Did the policies have an effect on climate?
	\item Which sectors remained unaffected by COVID-19?
	\item Which sectors where affected the most?
	\item Which measures are the most effective?
\end{itemize}