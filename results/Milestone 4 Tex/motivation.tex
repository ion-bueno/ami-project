\section{Introduction and Motivation}

% Task: [This section is about introduction, motivation and context. Draw a picture of the context within which your work is embedded. Describe what would be the ultimate long-term goal that you try to reach eventually]

% What we try to convey: "Research Question helps policy makers"

% Dramatic first sentence
Storm clouds gather over the world, mistrust and rivalries between countries threaten to break and flood the world in a tide of a global crisis.\footnote{Adapted from: \textit{Medieval 2: Total War. 2006. Developer: Creative Assembly. Publisher: Sega. Windows/macOS/Linux.}} Yet, countries started to work together an tackle the climate crisis. Just when we saw the first major changes and commitments, the COVID-19 pandemic stroke. Countries all over the world mainly focused on fighting back the virus, bringing climate crisis efforts to a halt. But politicians could possible learn from experiences made during the pandemic to tackle the climate crisis more effectively later. As a group of scientists and engineers, we see this as our job. We perform a thorough investigation of available data sets, make assumptions, predict values and critically assess our findings.

% State research question
Specifically, we want to investigate \textit{to what extent will the COVID-19 pandemic contribute towards reaching goals stated in international agreements}.

% Link to Paris Agreement
During our research on this question, we concluded that it is best to focus on the \textit{Paris Agreement}, as it gives a greater frame also to other international agreements. The \textit{Paris Agreement} got a lot of publicity when it got ratified by most countries of the world. It thus resembles the first global, legally binding agreement to fight global warming. When the US under president Trump decided to leave the agreement and climate activists started the \textit{Fridays for Future} movement, the problem of global warming got more attention world-wide. However, as the COVID-19 pandemic began, more urgent problems arose and global warming only came in second.

%Expectations
In the end, we want to see if we can say that countries with higher case numbers of COVID-19 also have less \co emissions and how this is related to one another.
Thus, the two most important fields of data we used for our model are COVID-19 and \co data. We did not have any problems regarding COVID-19 data but had some issues with \co data. Emission data is not up-to-date and is only published as yearly emissions per country. However, the COVID-19 pandemic influences emissions on a shorter time-scale as of now. We thus had to predict and model emission data ourselves.  We first predicted yearly \co emissions in a business-as-usual scenario without COVID-19 and then estimated \co emissions with COVID-19. To do so, we used industry data and developed indicators from that. We even did this for each of the five major sectors of \co emitters of every country individually. Similar approaches can also be found in the literature, for example in~\cite{LeQuere2020}.

%Ultimate long-term goal
First of all, we want to answer our research question and quantify how much the pandemic helped countries reaching their climate goals. With this, we also want to be able to adjust \co emission predictions if a second COVID-19 wave were to come.
However, our long-term goal is helping politicians to take effective measures, using findings based on experiences during the pandemic.